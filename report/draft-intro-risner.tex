\section{Introduction}
There have been several research projects at the University of Puget Sound concerned with bitmap indexes.
The advantages of bitmap indexes include their high compressibility which admits queries without decompression.
Further, these queries take advantage of the speed of bitwise operations in their processing.
The prior work on bitmap indexes has focused upon implementing several compression methods on a single CPU,
however due to the ever increasing size of data sets being used it is becoming increasingly impractical to confine data to a single machine;
this is where distributed systems come in.
By moving to a distributed system it becomes possible to provide computational speedup through parallelization, reliability through fault tolerance, and communication including the sharing of data (e.g., multiple research facilities sharing experimental results).
\par
We have created a framework around the existing bitmap engine that provides many of the benefits of a distributed system.
In our implementation we elected to use the master-slave model as it was highly compatible with our envisioned use case:
using the distributed system for index queries in a database management system.
In addition, for several components we decided to implement several possible solutions which are each more suited to a specific domain or use-case.
Such components include:
the partitioning method which can be static, based upon ring-consistent-hashing, or based upon jump-consistent-hashing \cite{lamping2014};
and multiple query planning methods such as UniStar, MultiStar, and IterPrims.
In addition, the system also has several configurable options, such as the data replication factor.
\par
With the conclusion of our work, the system provides a strong foundation for further development of a configurable index system taking advantage of the strengths of both bitmap indexes and distributed systems.
Further work could include supporting multiple bitmap compression methods (such as BBC and VAL) as the system currently only supports the use of WAH compression.
In addition, depending upon the deployment, the ability to dynamically create and remove nodes depending upon usage and load balancing would be of great usefulness.
