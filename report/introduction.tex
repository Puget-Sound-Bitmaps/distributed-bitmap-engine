\section{Introduction}
As datasets grow to be terabytes---or even petabytes---in size, performing
queries on them can become prohibitively time-consuming if not done
efficiently. An example of such a situation could be performing a query on US
census data (see Table~\ref{table:census-relational} for a small set of example
data). If we were to search for all Americans with salaries of at least
\SI{100000}[\$]{} who are also under the age of 50 (equivalent to the SQL query
\code{SELECT * FROM CENSUS WHERE SALARY~>~\$100000 AND AGE~<~50}), we would have to scan each
of the \num{326000000} lines (\term{tuples}) of the database sequentially, since each line corresponds
to a unique person. Each line read requires a separate disk access which can take upwards of \SI{15}{\milli\second},
even on fast computers. As a result, this query could take
\(\num{326 000 000} \times \SI{15}{\milli\second} = \SI{1400}{\hour}\)
which is just under two months; obviously queries cannot take this long.
\begin{table}[H]
    \centering
    \caption{Example of a Relational Database, \code{CENSUS}}
    \label{table:census-relational}
    \begin{tabular}{@{}r||rcll@{}}
        \toprule
        Tuple   & Salary (\$)  & Age & City    & Name  \\
        \midrule
        \(t_0\) & \num{65000}  & 20  & Tacoma  & Julia \\
        \(t_1\) & \num{25000}  & 76  & Spokane & Tim   \\
        \(t_2\) & \num{130000} & 42  & Seattle & Maria \\
        \bottomrule
    \end{tabular}
\end{table}
\par
Using a \term{bitmap index} is one method for improving query execution time. A
bitmap index is a collection of binary strings that represent truth values
pertaining to a relational database (see Tables~\ref{table:census-salary} and
\ref{table:census-age} for a possible index of the age and salary data in
Table~\ref{table:census-relational}).  In this paper we refer to each string as
a \term{bitmap vector}. Queries on a bitmap index can satisfy common
database queries---such as the SQL query given above---and are very efficient
as they consist primarily of machine-level bitwise operators. To satisfy the query,
we first find all Americans who make over \SI{100000}[\$]{} by \code{OR}ing
vectors \(v_2\) (Americans with salaries between \SI{100000}[\$]{} and
\SI{300000}[\$]{}) and \(v_3\) (Americans with salaries over \SI{300000}[\$]{}).
Second, we find all Americans under the age of 50 by \code{OR}ing bitmap vectors
\(v_4\), \(v_5\), and \(v_6\) which will include all Americans under the age of 66 (which is a superset of the tuples relevant to the query). After \code{AND}ing together the two ranges, the resulting bitmap vector will have a value of \code{1} in rows corresponding
to Americans who \emph{may} be included in the result of the SQL query. To find the exact query return value,
the original tuples corresponding to the rows with \code{1}s must be read.
While this may result in some tuples that do not satisfy the query being scanned,
the total number of tuples scanned is \emph{significantly} fewer than in a naive sequential scan.
\begin{table}[H]
    \centering
    \caption{Bitmap Index for Salary (\(S\), in thousands of dollars) in Table~\ref{table:census-relational}}
    \label{table:census-salary}
    \begin{tabular}{@{}r||c|c|c|c@{}}
        \toprule
                & \(S \leq 60\) & \(60 < S < 100\) & \(100 < S \leq 300\) & \(300 \leq S\) \\
                & \(v_0\) & \(v_1\) & \(v_2\) & \(v_3\) \\
        \midrule
        \(t_0\) & 0          & 1             & 0          & 0             \\
        \(t_1\) & 1          & 0             & 0          & 0             \\
        \(t_2\) & 0          & 0             & 1          & 0             \\
        \bottomrule
    \end{tabular}
\end{table}
%
\begin{table}[H]
    \centering
    \caption{Bitmap Index for Age (\(A\)) in Table~\ref{table:census-relational}}
    \label{table:census-age}
    \begin{tabular}{@{}r||c|c|c|c@{}}
        \toprule
                & \(A < 18\) & \(18 \leq  A < 21\) & \(21 \leq A < 66\) & \(66 \leq A\) \\
                & \(v_4\) & \(v_5\) & \(v_6\) & \(v_7\) \\
        \midrule
        \(t_0\) & 0          & 1             & 0          & 0             \\
        \(t_1\) & 0          & 0             & 0          & 1             \\
        \(t_2\) & 0          & 0             & 1          & 0             \\
        \bottomrule
    \end{tabular}
\end{table}
\par
A system comprising multiple computers (\term{nodes}) is known as a
\term{distributed system}. In comparison, a system containing only one node is
called a \term{centralized system}. There are two principal advantages of using a
distributed system over a centralized system: first, there is no
single point of failure, and second, the storage capacity of the system in
aggregate can reach a size infeasible for a centralized system. Should an
individual node in a distributed system fail, the data can be redistributed such
that there is again a backup of all data in the system. Such resilience to
hardware (or software) failure cannot be be obtained in a centralized system,
in which node failure would necessitate regenerating the entire index.
\par
The purpose of our system (hereafter called DBIE) is to expand upon the
functionality of Chiu, et al.'s bitmap engine by distributing a set of bitmap
vectors, and the work of executing bitmap queries, among multiple nodes. Our
system is designed to be capable of recovering from single-node failure at any
point during execution. Our approach was to handle vector input and query
requests from a database management system (DBMS) via a \term{master}
\term{process}. A process is a program in execution. The master and DBMS
processes run on the same node. When asked to store a vector, the master sends
replicas of the vector to two distinct nodes (\term{slaves}). To satisfy
queries, the master delegates work to the slaves that have the requisite
vectors.
\par
%
%
