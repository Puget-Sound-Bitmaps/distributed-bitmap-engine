\section{Results}
We have tested our implementation on a system comprising six slave nodes
against a centralized system making no RPCs. The distributed system was found
to take noticeably longer to handle a batch of bitmap queries of any given size
(\(0 \leq q \leq 1200\), where \(q\) is the number of queries) than the
centralized one. Figure~\ref{fig:graph-of-results} compares the execution time
of of the distributed system with that of the centralized system. It was found
that the distributed system had an execution time of
\(\mathrm{t}_d(q) = 4.52 \cdot 10^{-3} \cdot q + 7 \cdot 10^{-2}\)
with \(R^2 = 0.9919\) while the centralized system had an execution time of
\(\mathrm{t}_c(q) = 1.52 \cdot 10^{-4} \cdot q - 2.31 \cdot 10^{-3}\)
with \(R^2 = 0.9818\). In both cases, \(q\) is the number of queries, as stated
above, and the execution time is measured in seconds). Since the slope of
\(\mathrm{t}_d(q)\) surpasses the slope of \(\mathrm{t}_c(q)\), we expect the
distributed system to perform much more slowly. This is an expected result due
to the time required to perform RPCs and execute the query planner. Despite the
slower performance, using a distributed system provides fault tolerance.
Further, while a centralized system cannot be altered to provide fault
tolerance, the execution time of the distributed system could be improved
through further optimization.
%
\begin{figure}
    \centering
    % \includegraphics[width=\columnwidth]{results-graph}
    \includegraphics[width=\columnwidth]{query-experiment-results}
    \caption{Experimental Results}\label{fig:graph-of-results}
\end{figure}
